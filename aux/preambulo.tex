% abnTeX2: Modelo de Trabalho Academico (tese de doutorado, dissertacao de
% mestrado e trabalhos monograficos em geral) em conformidade com 
% ABNT NBR 14724:2011: Informacao e documentacao - Trabalhos academicos -
% Apresentacao

% CONFIGURAÇÕES DO DOCUMENTO

\documentclass[
	12pt, % tamanho da fonte
	openright, % capítulos começam em pág ímpar 
	twoside, % impressão frente e verso (oposto a oneside)
	a4paper, % tamanho do papel. 
	english, % idioma adicional para hifenização
	french, % idioma adicional para hifenização
	spanish, % idioma adicional para hifenização
	brazil, % o último idioma é o principal do documento
	table, xcdraw % opções de xcolor para colorir tabelas
	]{abntex2}

% PACOTES BÁSICOS

\usepackage{times} % Usa a fonte Times New Roman
\usepackage[T1]{fontenc} % Selecao de codigos de fonte.
\usepackage[utf8]{inputenc} % Codificação de acentos
\usepackage{indentfirst} % Indenta o primeiro parágrafo 
\usepackage{color} % Controle das cores
\usepackage{graphicx} % Inclusão de gráficos
\usepackage{microtype} % Para melhorias de justificação

% PACOTES ADICIONADOS PELO AUTOR

\usepackage{amsmath} % Escrita matemática
\usepackage{amssymb} % Escrita matemática
\usepackage[textsize=tiny, disable]{todonotes} % Adição de notas 
\usepackage{booktabs} % Inserção de tabelas com estilo de livro
\usepackage[font=small]{caption} % Inserção de legendas em minipages
\usepackage{icomma} % Correção da distância entre a vírgula e os números decimais
\usepackage{bm} % Emprego de símbolos negritos em equações
\usepackage{nccmath} % Controle do tamanho de frações dentro de frações
\usepackage{gensymb} % Inserção do símbolo de grau
\usepackage{tabu} % Criação de tabelas com quebra de linha automática

% COMANDOS ADICIONADOS PELO AUTOR

\setcounter{MaxMatrixCols}{20} % Delimita o número máximo de colunas em uma matrix

\raggedbottom % Previne que o texto se expanda pra preencher a página

% Definição do ponto para produto escalar
\makeatletter
\newcommand*\bigcdot{\mathpalette\bigcdot@{.7}}
\newcommand*\bigcdot@[2]{\mathbin{\vcenter{\hbox{\scalebox{#2}{$\m@th#1\bullet$}}}}}
\makeatother

% PACOTES PARA CITAÇÕES

\usepackage[brazilian,hyperpageref]{backref} % Hipertexto nas referências
\usepackage[alf, bibjustif]{abntex2cite} % Citações no padrão ABNT

% CONFIGURAÇÃO DO PACOTE backref

% Usado sem a opção hyperpageref de backref
\renewcommand{\backrefpagesname}{Citado na(s) página(s):~}
% Texto padrão antes do número das páginas
\renewcommand{\backref}{}
% Define os textos da citação
\renewcommand*{\backrefalt}[4]{
	\ifcase #1 %
		Nenhuma citação no texto.%
	\or
		Citado na página #2.%
	\else
		Citado #1 vezes nas páginas #2.%
	\fi}%
\renewcommand{\ABNTEXchapterfont}{\scshape\fontfamily{cmr}\fontseries{b}\selectfont}

% INFORMAÇÕES PARA CAPA E FOLHA DE ROSTO

\titulo{Estudo comparativo de pacotes computacionais empregados na resolução de problemas de Controle Ótimo}
\autor{Arthur Henrique Iasbeck}
\local{Uberlândia, MG}
\data{2021}
\orientador{Fran Sérgio Lobato}
\instituicao{%
  Universidade Federal de Uberlândia (UFU)
  \par
  Faculdade de Engenharia Mecânica
  \par
  Programa de Pós-Graduação em Engenharia Mecânica}
\tipotrabalho{Dissertação (Mestrado)}
\preambulo{\textbf{Dissertação} apresentada ao Programa de Pós-graduação em Engenharia Mecânica da Universidade Federal de Uberlândia como parte dos requisitos para obtenção do título de \textbf{Mestre em Engenharia Mecânica}. \newline \newline \textbf{Área de Concentração}: Mecânica dos Sólidos e Vibrações. \newline \textbf{Linha de Pesquisa}: Projetos de Sistemas Mecânicos.}

% CONFIGURAÇÕES GERAIS

% alterando o aspecto da cor azul
\definecolor{blue}{RGB}{41,5,195}

% informações do PDF
\makeatletter
\hypersetup{
     	%pagebackref=true,
		pdftitle={\@title}, 
		pdfauthor={\@author},
    	pdfsubject={\imprimirpreambulo},
	    pdfcreator={LaTeX with abnTeX2},
		pdfkeywords={abnt}{latex}{abntex}{abntex2}{trabalho acadêmico}, 
		colorlinks=true, % false: links em caixas; true: links coloridos
    	linkcolor=blue, % cor dos links internos
    	citecolor=blue, % cor dos links da bibliografia
    	filecolor=magenta, % cor dos links para arquivos
		urlcolor=blue,
		bookmarksdepth=4
}
\makeatother

% Posiciona figuras e tabelas no topo da página quando adicionadas sozinhas em um página em branco
\makeatletter
\setlength{\@fptop}{5pt} % Set distance from top of page to first float
\makeatother

% Possibilita criação de Quadros e Lista de quadros.
\newcommand{\quadroname}{Quadro}
\newcommand{\listofquadrosname}{Lista de quadros}

\newfloat[chapter]{quadro}{loq}{\quadroname}
\newlistof{listofquadros}{loq}{\listofquadrosname}
\newlistentry{quadro}{loq}{0}

% Configurações para atender às regras da ABNT
\setfloatadjustment{quadro}{\centering}
\counterwithout{quadro}{chapter}
\renewcommand{\cftquadroname}{\quadroname\space} 
\renewcommand*{\cftquadroaftersnum}{\hfill--\hfill}

\setfloatlocations{quadro}{hbtp}

% ESPAÇAMENTO ENTRE LINHAS E PARÁGRAFOS

% O tamanho do parágrafo 
\setlength{\parindent}{1.3cm}

% Controle do espaçamento entre um parágrafo e outro
\setlength{\parskip}{0.2cm}  % tente também \onelineskip

% Compila o indice
\makeindex