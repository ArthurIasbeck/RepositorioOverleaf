Nesse capítulo, os  pacotes $ COPILOTS $, $ PSOPT $ e $ FALCON $ serão avaliados considerando os seguintes estudos de caso:
%
\begin{itemize}
\item Minimização do esforço durante a aceleração de um bloco \cite{becerra_optimal_2008};
\item Problemas singulares: Casos 1 e Caso 2 \cite{jacobson_computation_1970};
\item Minimização do esforço durante o \textit{Swing-up} de um pêndulo invertido \cite{kelly_introduction_2017};
\item Minimização do tempo durante uma manobra de estacionamento \cite{li_time-optimal_2016};
\item Otimização da trajetória de um UAV (\textit{Unmanned Aerial Vehicle}) \cite{toledo_de_azevedo_pseudospectral_2018};
\item Lançamento do foguete Delta III \cite{benson_gauss_2005}.
\end{itemize}

Cabe ressaltar que, conforme discutido anteriormente, estes foram escolhidos por apresentarem diferentes níveis de complexidade. Sendo assim, pode-se dizer que estes reúnem boas características para a validação da metodologia proposta neste trabalho, bem como para avaliar os outros pacotes considerados. 

No decorrer desse capítulo cada um dos estudos de caso são apresentados, formulados matematicamente, as respectivas soluções obtidas são apresentadas e a comparação entre os pacotes é realizada. Ao término deste capítulo é apresentado um consolidado dos resultados obtidos.