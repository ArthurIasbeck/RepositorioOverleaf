Nessa seção são apresentadas algumas questões que devem ser respondidas pelo usuário que pretende escolher o pacote mais adequado. Alguns comentários acerca dos métodos avaliados no presente trabalho são elaborados, porém, é necessário que se tenha em mente que as questões aqui apresentadas podem servir de base para comparação de pacotes que não foram aqui avaliados, ou mesmo que não tenham sido desenvolvidos para resolução de PCOs \cite{parejo_metaheuristic_2012}. 

\begin{itemize}
	\item \textbf{O pacote exige uma licença paga? }
	
	Nenhum dos pacotes avaliados exige uma licença paga. Vale ressaltar que apesar do $ FALCON $ possuir uma versão paga, foi aqui avaliada a versão gratuita desse pacote, que deve ser solicitada aos desenvolvedores.
	
	\item \textbf{O pacote possui código fonte aberto? }
	
	Tanto o $ COPILOTS $ quanto o $ PSOPT $ possuem código fonte aberto, enquanto o $ FALCON $ não.
	
	\item \textbf{O pacote executa em quais plataformas (Linux, Windows\textsuperscript{\textregistered}, Mac\textsuperscript{\textregistered}, etc...)?}
	
	O $ PSOPT $ executa apenas no Linux. Já o $ COPILOTS $ e o $ FALCON $ são baseados no Matlab\textsuperscript{\textregistered}, o que significa que podem ser executados em qualquer plataforma. 
	
	\item \textbf{Quantos exemplos acompanham o código fonte do pacote?}
	
	O $ PSOPT $, o $ FALCON $ e o $ COPILOTS $ trazem em sua documentação 45, 11 e 7 exemplos, respectivamente.
	
	\item \textbf{A documentação do pacote é completa e fornece as informações necessárias para que o usuário resolva os estudos de caso que lhe interessam? A documentação do pacote fornece a base teórica necessária ao entendimento dos métodos no qual o pacote se baseia?}
	
	Nesse quesito, o $ PSOPT $ é o pacote que mais se destaca, tendo em sua documentação uma introdução bastante completa acerca da colocação pseudo-espectral. Já a documentação do $ FALCON $ traz instruções que possibilitam a sua utilização sem grandes dificuldades, mas carece de explicações mais detalhadas acerca de muitos dos recursos de que o pacote dispões. Já o $ COPILOTS $ não possui documentação alguma, visto que é um pacote extremamente novo.  
	
	\item \textbf{Quantos trabalhos foram desenvolvidos empregando-se o pacote? }
	
	O $ PSOPT $ se destaca mais uma vez nesse quesito, uma vez que no site oficial desse pacote estão listados 60 trabalhos que empregaram o $ PSOPT $ de alguma forma, desde artigos em periódicos e conferências, até livros e teses. São citados no site do $ FALCON $ poucos trabalhos que fazem uso do pacote, porém, é possível encontrar na literatura especializada cerca de 30 trabalhos que empregam o $ FALCON $ de alguma forma. Já o $ COPILOTS $ serviu de base para o desenvolvimento de apenas 2 trabalhos, já que é um pacote extremamente novo. 
	
	\item \textbf{O pacote possui uma comunidade ativa de usuários? }
	
	Apenas o $ PSOPT $ possui uma comunidade ativa de usuários que se comunica por meio do \textit{Google Groups}.
	
	\item \textbf{O pacote possui suporte por parte dos desenvolvedores?}
	
	Nesse quesito, o $ FALCON $ se destaca, já que em linhas gerais, é um pacote pago. Contactando os desenvolvedores por e-mail, é possível que o usuário tenha suas dúvidas sanadas muito brevemente, em alguns casos em questão de horas.
	
	\item \textbf{Com que frequência o código fonte do pacote recebe atualizações?}
	
	O $ PSOPT $ recebe atualizações esporádicas, tendo sido a versão 5.0 lançada cerca de um ano e meio após o lançamento da versão 4.0. Nem o site do $ FALCON $ nem a documentação associada ao mesmo indicam quanto tempo se passou desde o lançamento da última versão do pacote. Já o $ COPILOTS $ nunca foi atualizado.   
	
	Vale ressaltar que é possível que pacotes que tenham sido  podem sofrer atualizações com uma frequência demasiadamente alta. Dessa forma, é possível que uma aplicação se torne obsoleta muito rapidamente, dependendo de quão profundas forem as modificações trazidas por essas atualizações. Assim sendo, recomenda-se o uso de pacotes que já tenham se estabelecido perante a comunidade acadêmica e que já sejam amplamente utilizados. 
	
	\item \textbf{O pacote é de fácil utilização?}
	
	Vale aqui ressaltar que a resposta a essa pergunta depende diretamente do usuário que irá utilizar o pacote. É recomendado que a documentação associada ao mesmo seja avaliada assim como os exemplos que acompanham o código fonte. 
	
	\item \textbf{ É possível obter bons resultados empregando-se o pacote?}
	
	Essa pergunta não pode ser respondida facilmente, porém, uma das formas de respondê-la é realizando análises semelhantes às que foram introduzidas no decorrer do presente capítulo. 
	
\end{itemize}