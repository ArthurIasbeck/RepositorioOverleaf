O Problema de Controle Ótimo (PCO) consiste na determinação do vetor de variáveis de controle para a minimização de uma função objetivo sujeito à restrições algébrico-diferenciais. De forma geral, este problema pode ser resolvido considerando dois tipos de abordagens, a saber, a Direta e a Indireta. A primeira consiste na transformação do problema original em um equivalente de programação não linear. Já a abordagem Indireta consiste na aplicação das condições de otimalidade transformando o problema original em um equivalente problema de valor no contorno algébrico-diferencial. Devido a dificuldade inerente em resolver este problema de valor no contorno, a comunidade científica tem utilizado, preferencialmente, os Métodos Diretos para a resolução de PCO. Neste contexto, na literatura especializada podem ser encontrados vários pacotes numéricos que se fundamentam na abordagem Direta para a resolução de PCOs. Apesar da ampla variedade de pacotes disponíveis, poucos são os trabalhos que avaliam a eficiência e as características destes. Diante do que foi apresentado, este trabalho tem por objetivo apresentar um estudo comparativo considerando alguns solvers para a resolução de PCOs. Para essa finalidade foi desenvolvido o pacote COPILOTS que implementa os Métodos de Colocação Trapezoidal e de Hermite-Simpson para a resolução de PCO. Os resultados obtidos demonstram que o COPILOTS foi capaz de obter bons resultados em termos do valor da função objetivo, número de avaliações da função objetivo e número de pontos de colocação em comparação com os solvers PSOPT e FALCON. Finalmente, é importante ressaltar que o COPILOTS é um pacote desenvolvido para usuários com pouca experiência em controle ótimo, de fácil implementação e uso.

\textbf{Palavras-chave}: Controle Ótimo. Métodos Diretos. Estudo Comparativo. Pacotes Computacionais.
